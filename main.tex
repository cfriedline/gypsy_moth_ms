\documentclass[fleqn,11pt]{wlpeerj}
\usepackage{lineno}

\title{The ecological genomics of gypsy moth invasion along a latitudinal
gradient}

\author[1]{Christopher J. Friedline}
\author[1]{Trevor M. Faske}
\author[1]{Erin M. Hobson}
\author[2]{Dylan Parry}
\author[1]{Derek M. Johnson}
\author[3]{Lily M. Thompson}
\author[3,*]{Kristine L. Grayson}
\author[1,*,\textdagger]{Andrew J. Eckert}
\affil[1]{Department of Biology, Virginia Commonwealth University}
\affil[2]{College of Environmental Science and Forestry, State University of
New York}
\affil[3]{Department of Biology, University of Richmond}
\affil[*]{Author contributed equally}
\affil[ \textdagger]{Corresponding author}

\keywords{}

\begin{abstract}
TODO
\end{abstract}

\begin{document}

\flushbottom
\maketitle
\thispagestyle{empty}

\linenumbers

\section*{Introduction}
TODO


\section*{Methods}

\subsection*{Library preparation and sequencing}

Genomic DNA extracted using Qiagen DNeasy Plant Mini kit (Qiagen, cat \# 69104)
according to manufacturer's protcool.  The library was prepared and sequenced as
a single run of paired-end reads on the Illumina HiSeq 2500 platform at the
Virginia Commonwealth Univesrity Nucleic Acids Research Facility. Quality was
initially asssed using FastQC \citep{fastqc} and subsequently processed with
IPython \cite{Perez:2007hy} and BioPython \cite{Cock:2009hj}. The read pairs
were each evaluated along sliding windows of 5 bp.  If the mean score in this
window was below a score of 30, the read was trimmed at that beginning of the
window. If the shortened read was less than 50\% of the length of the original
read, it was discarded along with its pair. Additionally, if 20\% of the bases
in a read had quality values less than 30, it was discarded along with its pair.

\subsection*{Draft assembly}

An assembly was produced for the sole purpose of variant calling and is not
considered as a candidate for complete genomic assembly. The quality-filtered
reads from the paired-end sequencing library was used as input into
\texttt{MaSuRCA} \citep{Zimin:2013kn}, version 2.3.2, with the following
parameters changed from the defaults: mean/standard deviation of read length:
400/60, use linking mates: 1: cgwErrorRate: 0.15.  Quality of the assembly was
assessed using the stats generated by the assembly process as well as with the
\texttt{BUSCO} \citep{Simao:2015kk} toolset. For \texttt{BUSCO}, the software
was built automatically using a Docker image (see GitHub repository information
below). The assembly was evaluated against pre-computed \texttt{Augustus}
\citep{Stanke:2003eo} metaparameters of  three species (\textit{Aedes},
\textit{Heliconius}, and \textit{Drosophila}) and the BUSCO Arthropoda lineage
profile using a \texttt{tblastn} e-value cutoff of $0.001$.

\subsection*{Sequence processing and variant calling} Sequencing of 192
individual moths, exclusive of the one used for assembly,  as performed on
reduced genomic DNA per \cite{PARCHMAN:2012ca}. In total, two  Illumina HiSeq
2500 lanes were sequenced. Bases with quality scores less than 33 were masked,
per \cite{Yun:2014dn},  using \texttt{fastq\_masker} from the FastX toolkit
\citep{citeulike:9103573}, version 0.0.14.

Masked fastq files were mapped using \texttt{bowtie2} \cite{Langmead:2012jh},
version 2.2.4, using flags \texttt{--local --very-sensitive-local}.  The
resulting  sam files were converteted to their binary equivalent, sorted, and
indexed. To each of these indexed bam files, appropriate \texttt{RG} ID and
sample information was added using \texttt{picard} \\
(\url{https://broadinstitute.github.io/picard}) version 1.112 to inform variant
calling.

Duplicates were marked using \texttt{Picard} and the resulting files were
indexed using \texttt{samtools} \citep{Li:2009ka}, version 1.2.  Sequence
variants were called using samtools 1.2 and bcftools 1.2 with individual  ploidy
level set to \texttt{2} for all samples.  The variants were filtered  using
\texttt{vcftools} \cite{Danecek:2011gz}, version 0.1.14, such that only
biallelic-SNPs remained in cases where a variant call was present in at least
50\% of the samples and the minimum gentoype quality (\texttt{--minQG}) of 3. An
additional filtering using \texttt{vcfutils.pl} (from \texttt{bcftools}) was
performed to only keep sites with mapping qualities of at least 10
(\texttt{varFilter -Q 10}). The vcf file was then double checked to ensure all
reference and alternate alleles were single nucleotides before imputation with
\texttt{Beagle} \citep{Browning:2007ge}, version 4.0.  The impuation was
performed on 30 CPUs using the following non-default parameters
(\texttt{phase-its=10, burnin-its=20, impute-its=10}).

For each set of imputed (IMP) and non-imputed (NI) SNPs, the following filtering
was performed, in order: monomorphic SNPs were removed, all SNPs with minor
allele frequencies (corrected for finite population size) $< 0.01$ were removed,
SNPs with $F_{IS} < -0.05$ or $F_{IS} > 0.5$ were removed.  Finally, SNPs were
oriented  according to dosage of globally-minor allele, rather than alternate.
In other words,  in a file of genotypes where 0, 1, 2 would normally represent
counts of alternate allele  (as output by \texttt{vcftools}), 0,1,2 were recoded
to represent counts of the  minor allele.  This ``swapped" representation was
then used as input for  downstream analysis

\subsection*{Population statistics}

\subsection*{Environmental assocation}

\subsection*{Loci under selection}


\subsection*{}

\section*{Results}
TODO


\section*{Discussion}
TODO


\section*{Conclusion}
TODO


\section*{Figures}
TODO

\section*{Tables}
TODO

\section*{Supplementary material}
TODO

\clearpage
\bibliography{refs}

\end{document}
